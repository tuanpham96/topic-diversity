\section{Introduction}

As the world becomes more and more connected, as well as information increases more and more every day, how do we learn about the currently existing body of knowledge, and at the same time updating with the newer incoming information? Are people becoming more specialized or are there more generalists? Answering these questions might be difficult at this point without assessing simple cases of learning within static networks. Hence I want to examine how different knowledge acquistion strategies could affect one's knowledge set, as well as the diversity of knowledge for the population as a whole.

There could be multiple ways a person could learn something new, for example consider a PhD student reading a paper. In an overly simplified view, the student could start digging down the references for something they have never seen or heard about before and choose to actually read passionately about it, and let's optimistically assume they master it. Further and further down the    ``rabbit hole", the student starts learning about some of the most arcane, obscure subjects in human's knowledge (\autoref{fig:1}b).

In another case, the student goes to classes or discusses with their friends about a certain topic, or seeing new papers recommended through the people they follow on social media with the hashtag ``\#TheNextBigThingIn[\textit{insert-field}]''. Through these interactions, they could pick up on something that are new to them and start broadening their horizons, based on their peers' recommendations (\autoref{fig:1}c).

Considering only these two different ways of learning in a probabilistic sense, I examine the diversity of knowledge, represented as different metrics based on the distribution of topics, as well as graph metrics, in random networks, with and without consideration of modularity within such networks. The results show that the self-learning process tends to improve diversity in a population manner, but the latter process by learning through friends or recommendations would generally benefit individual diversity. Consideration of groups within the models have mixed effects at the individual level more so than the population level.